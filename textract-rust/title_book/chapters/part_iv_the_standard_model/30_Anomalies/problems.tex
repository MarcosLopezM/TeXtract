% Archivo generado automáticamente con los problemas
\section*{Problems}
Sección: 30_Anomalies
Páginas: 659-659
Contenido:
30.1 Baryon number has an anomaly so that ∂μJB
μ ̸= 0 as in Eq. (30.88). Since the right-
hand side of Eq. (30.88) has more than two gauge fields, it implies that diagrams
such as
with the
indicating JB
μ (x), should also give non-zero answers when contracted
with ∂μ. Evaluate this diagram and any other that contributes at the same order to
show that the W 3 terms in Eq. (30.88) are correctly reproduced.

30.2 For which types of neutrino masses (Majorana, Dirac or both) is lepton number
anomalous? For which types of masses is B −L anomalous?

30.3 Suppose that QCD were based on the gauge group SU(5). Let us assume that the
proton still exists as a five-quark bound state with charge +1, so that quarks now have
five colors and electric charges in Z/5. What values for the SU(5) × SU(2)weak ×
SU(1)Y quantum numbers of the Standard Model fields would make this universe
anomaly free?

30.4 Can anomaly matching arguments determine if SU(4)L × SU(4)R is spontaneously
broken in QCD?


---

