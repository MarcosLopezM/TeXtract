% Archivo generado automáticamente con los problemas
\section*{Problems}
Sección: 28_Spontaneous_symmetry_breaking
Páginas: 602-602
Contenido:
28.1 Show that writing φ(x) =

2m2
λ + ˜φ(x) for the linear sigma model in Section 28.2.1
leads to a mass matrix with zero eigenvalue. Show that when a linear combination of
the two real fields in the complex field ˜φ is chosen to diagonalize the mass matrix,
the expansion in Eq. (28.12) results.

28.2 Work out the transformations to order π2 and θ2 in Eq. (28.25) using the Baker–
Campbell–Hausdorff lemma:
exp(A) exp(B)
= exp
A + B + 1
2[A, B] + 1
12[A, [A, B]] −1
12[B, [A, B]] + · · ·
. (28.72)
Show that pions transform in the adjoint representation under isospin.

28.3 Work out the interaction terms of order π3 in the gauged nonlinear sigma model in
Eq. (28.59).

28.4 Consider a theory with n real scalar fields and Lagrangian L = −1
2φi(□−m2)φi +
λ
4 (φiφi)2.
(a) What are the global symmetries of this theory?
(b) What are all the possible vacua of this theory? Are all the vacua equivalent?
(c) Write down the Lagrangian for small excitations around one of the vacua. How
many Goldstone bosons are there?

28.5 For grand unification based on SU(5) to work, there must be a potential for the 24
scalar fields Φa such that Φ = Φaτ a has a minimum in the form of Eq. (28.55).
Consider the most general SU(5)-invariant potential for Φ:
V = −m2tr(Φ2) + a tr(Φ4) + b

tr(Φ2)
2 .
(28.73)
One can always choose a basis where ⟨Φ⟩
=
v diag(a1, a2, a3, a4, a5) with

i ai = 0.
(a) For what values of m2, a and b is ⟨Φ⟩= v diag(2, 2, 2, −3, −3) an extremum?
(b) Show that excitations around ⟨Φ⟩= v diag(2, 2, 2, −3, −3) all have non-
negative mass-squared.
(c) Find all possible minima for this potential. This is easiest if you impose the
tracelessness condition with a Lagrange multiplier.
(d) For the minimum of the form ⟨Φ⟩= v diag(1, 1, 1, 1, −4), what are the masses
of the massive gauge bosons, and what is the unbroken gauge group?


---

