% Archivo generado automáticamente con los problemas
\section*{Problems}
Sección: 35_Heavy-quark_physics
Páginas: 794-794
Contenido:
35.1 Reparametrization invariance.
(a) Show that the HQET Lagrangian including the leading m−1
Q
corrections, Eq.
(35.72), is invariant under
vμ →vμ +
1
mQ
kμ,
Qν →eik·x
1 +
/k
2mQ
Qv,
(35.81)
with v·k = 0 and k ≪mQ. This transformation is known as reparametrization
invariance. It corresponds to the arbitrariness in the choice of vμ.
(b) Use reparametrization invariance to show that the Qv
D2
2mQ Qv term in the HQET
Lagrangian cannot be renormalized separately from the Qvv · DQv term.
(c) Confirm through a direct 1-loop calculation that these two terms are indeed
renormalized in the same way.

35.2 Calculate the anomalous dimension of the HQET operator
g
4mQ QvσμνQvF μν at
1-loop.


---

