% Archivo generado automáticamente con los problemas
\section*{Problems}
Sección: 15_The_Casimir_effect
Páginas: 318-318
Contenido:
15.4.2 Interpretation of counterterms
Another way of getting the same result is to add a counterterm to the Lagrangian. That
means adding another interaction that is just like the first, but infinite. So we take as our
Lagrangian
L = −1
2φ□φ −λR
4! φ4 −δλ
4! φ4,
(15.70)
where the counterterm δλ is infinite, but formally of order λ2
R. Then, working to order λ2
R,
the amplitude is
M(s) = −λR −δλ −λ2
R
32π2 ln s
Λ2 + O(λ4
R).
(15.71)
Now we can choose δλ to be whatever we want. If we take it to be
δλ = −λ2
R
32π2 ln s0
Λ2 ,
(15.72)
then
M(s) = −λR + λ2
R
32π2 ln s
s0
,
(15.73)
which is finite. In particular, this choice of δλ makes M(s0) = −λR, which was our
definition of λR above.
Doing things this way, with counterterms but as a perturbative expansion in the physical
coupling λR, is known as renormalized perturbation theory. The previous way, where we
compute physical quantities such as M(s1)−M(s2) directly, is sometimes called physical
or on-shell perturbation theory. The two are equivalent, but for complicated calculations,
renormalized perturbation theory is often much easier.
Problems

15.1 Evaluate the Casimir force using the Gaussian regulator in Eq. (15.29).

15.2 Show that the Casimir force from the vacuum energy of fermions has the opposite
sign than from bosons.

15.3 It has been proposed that geckos use the Casimir force to climb walls. It is known
that geckos do not use suction (like salamanders) or capillary adhesion (like some
frogs). A gecko’s foot is covered in a million tiny hairs called setae, which terminate
in spatula-shaped structures around 0.5 µm wide. Use dimensional analysis and the
form of the Casimir force to decide whether you think this could be possible.

15.4 The vacuum energy of massive particles also contributes to the Casimir force. Before
doing the calculation, how do you expect the Casimir force to depend on mass? Now
do the calculation and see if you are correct (use any approximations you want – this
problem is challenging).


---

