% Archivo generado automáticamente con los problemas
\section*{Problems}
Sección: 21_Renormalizability
Páginas: 412-412
Contenido:
21.2.4 Summary
In summary:
• Renormalizable theories require only a finite number of counterterms.
• Non-renormalizable theories require an infinite number of counterterms.
• To renormalize non-renormalizable Lagrangians we must include every term not
forbidden by symmetries.
• Non-renormalizable theories can be renormalized. After renormalization all Green’s
functions are UV finite.
• Non-renormalizable theories are predictive at loop level, particularly through non-
analytic dependence on external momenta.
From a practical point of view, having a finite number of counterterms and renormalization
conditions is a huge advantage. Nevertheless, non-renormalizable theories are still very
predictive, often more so than renormalizable ones. We discuss these issues further in the
next chapter through a number of Standard Model examples. Non-renormalizable theories
play a central role in Part IV and especially Part V of this book.
Problems

21.1 Write down all the superficially divergent amplitudes in QED at 2-loops. Prove that
all of the UV divergences can be removed with the same four counterterms required
to remove the 1-loop divergences.

21.2 Calculate the contributions of
⃗p 4
M 4 ,

⃗p 2
M 2 and ln2 ⃗p 2
M 2 to a potential V(r) by taking
their Fourier transforms. Which gives the strongest contribution to the potential at
large distances? Which gives the weakest contribution?

21.3 Write down all the renormalizable interactions for a field theory with a single scalar
field φ(x) in two, three, four, five and six dimensions.


---

