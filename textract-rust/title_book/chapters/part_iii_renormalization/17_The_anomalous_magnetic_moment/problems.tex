% Archivo generado automáticamente con los problemas
\section*{Problems}
Sección: 17_The_anomalous_magnetic_moment
Páginas: 340-340
Contenido:
17.1 In supersymmetry, each fermion has a scalar partner, and each gauge boson has a
fermionic partner. For example, the partner of the electron is the selectron (˜e), the
partner of the muon is the smuon (˜μ), and the partner of the photon is the photino
( ˜A). The Lagrangian gets additional terms:
LSUSY = LSM + 1
2(∂μ˜e + igAμ˜e)(∂μ˜e + igAμ˜e) + m2
˜e˜e2 + g˜ee ˜A
+ ˜A(/∂+ m ˜
A) ˜A + 1
2(∂μ˜μ + igAμ˜μ)(∂μ + igAμ˜μ) + m2
˜μ˜μ2 + g˜μμ ˜A.
(17.33)
The smuon and selectron have the same electric charge, −1 (here g denotes the
electric charge, αe = g2
4π ∼
1
137). The size of the Yukawa couplings is fixed to be g
as well, by gauge invariance and supersymmetry.
(a) Calculate the contribution of loops involving the smuon to the muon’s magnetic
dipole moment.
(b) The current best experimental value for g −2 of the muon is
gμ−2
2
=
11 659 208.0 ± (6.3 × 10−10). The current theory prediction (assuming the
Standard Model only) is gμ−2
2
= 11 659 182.0 ± (8.0 × 10−10). What bound on
m˜μ do you get from this measurement?
(c) For other reasons, we expect m ˜
A ∼m˜μ ∼m˜e ∼MSUSY. What bound on
MSUSY do you get from the muon g −2?


---

