% Archivo generado automáticamente con los problemas
\section*{Problems}
Sección: 16_Vacuum_polarization
Páginas: 333-333
Contenido:
16.1 Calculate the pμpν pieces of the vacuum polarization graph in scalar QED and in
spinor QED. Show that your result is consistent with the Ward identity.

16.2 Calculate the Uehling potential, Eq. (16.60), by Fourier transforming the effective
potential.

16.3 The pions, π±, are charged scalar quark–antiquark bound states (mesons) with
masses of 139 MeV. The tauon is a lepton with mass 1770 MeV. Consider the con-
tribution of the vacuum polarization amplitude to π+π−→π+π−through a virtual
τ loop in QED. For simplicity, consider the s-channel contribution only.
(a) Plot |M|2 as a function of s for forward scattering (t = 0). You should find a
kink at s = s0. What is s0? What is going on physically when s > s0?
(b) Plot the real and imaginary parts of M separately. Calculate Im(M) explicitly
and show that it agrees with your plot.
(c) Find a relationship between Im(M) at t = 0 and the total rate for π+π−→
e+e−. This is a special case of a general and powerful result known as the optical
theorem, which is discussed in detail in Chapter 24.

16.4 Where is the location of the Landau pole in QED if you include contributions from
the electron, muon and tauon (all with charge Q = −1), from nine quarks (three
colors times three flavors) with charge Q =
2
3 and from nine quarks with charge
Q = −1
3?


---

