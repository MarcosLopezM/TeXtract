% Archivo generado automáticamente con los problemas
\section*{Problems}
Sección: 22_Non-renormalizable_theories
Páginas: 435-435
Contenido:
22.1 Treating the
p4
m4 term in the Schr¨odinger equation as a perturbation, calculate its
effects on the energy levels of the hydrogen atom in quantum mechanics. Compare
your result to the effect of a ln ⃗p 2
m2 term. Which can be more easily measured?

22.2 Calculate the term of order M −4 in the expansion of the 4-Fermi theory. That is,
expand Eq. (22.15) as in Eq. (22.16) to next order. You can use that the spinors
are on-shell, but you should not have factors of momentum or s – any factors of
momentum must come from derivatives acting on the spinor fields.

22.3 Verify the coefficients in Eq. (22.20). Write down one correctly normalized term in
the expansion of each term in Eq. (22.21).

22.4 In a scalar approximation to gravity, show that an interaction of the form
L1
1
MPl h□2h2, as in Eq. (22.30), indeed generates an exponentially suppressed
contribution to Newton’s potential, as in Eq. (22.33).

22.5 What is the form of the non-relativistic potential in a theory with a gφ3 interaction?
Why might this theory have been considered relevant as a possible theory of strong
interactions in the 1960s?


---

