% Archivo generado automáticamente con los problemas
\section*{Problems}
Sección: Appendix_B_Regularization
Páginas: 852-852
Contenido:
B.4.3 Other regulators
There are several other regulators that are sometimes used:
• Hard cutoff: kE < Λ. This breaks Lorentz invariance, and usually every symmetry in
the theory, but is perhaps the most intuitive regularization procedure.
• Point splitting. Divergences at k →∞correspond to two fields approaching each other
x1 →x2. Point splitting puts a lower bound on this, |xμ
1 −xμ
2| >|ϵμ|. This also breaks
translation invariance and is impractical for gauge theories, but is useful in theories with
composite operators.
• Lattice regularization. Although a lattice breaks both translation invariance and Lorentz
invariance, it is possible to construct a lattice such that translation and Lorentz invariance
are restored in the continuum limit (see Section 25.5).
Problems

B.1 Show that the Wick rotation still works if Δ < 0.


---

