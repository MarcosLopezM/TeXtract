% Archivo generado automáticamente con los problemas
\section*{Problems}
Sección: 12_Spin_and_statistics
Páginas: 242-242
Contenido:
12.1 In a causal theory, commutators of observables should vanish outside the light-
cone, [φ(x), φ(y)] = 0 for (x −y)2 < 0. For spinors, we found that with
anticommutation relations
+ ¯ψ(x), ψ(y)
,
= 0 outside the lightcone. This implies
that integer spin quantities constructed out of spinors are automatically causal,
e.g.
 ¯ψψ(x), ¯ψψ(y)

= 0. However, this is not a proof that spinors must anti-
commute. What would happen to
 ¯ψψ(x), ¯ψψ(y)

outside the lightcone if we
used commutation relations for spinors? For simplicity, you can just look at
0|
 ¯ψψ(x), ¯ψψ(y)

|0

.


---

